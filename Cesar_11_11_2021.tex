\documentclass[fleqn]{article}
\usepackage[left=1in, right=1in, top=1in, bottom=1in]{geometry}
\usepackage[utf8]{inputenc}
\usepackage[spanish]{babel}
\usepackage{mathexam}
\usepackage{amsmath}
\usepackage{amsfonts}
\usepackage{graphicx}
\usepackage{multirow}

\ExamClass{Topología I}
\ExamName{AMAT}
\ExamHead{11 de noviembre de 2021}

\let\ds\displaystyle

\begin{document}
\ExamInstrBox{Manuel César Rosales Lombardo
	\\ \\
	Parcial - Grupo B
	\\ \\
	Este examen pertenece al Banco de Exámanes de la Asociación de Estudiantes de Matemáticas de la Universidad de Granada. Si bien su autoría corresponde a los profesores ya citados, en la asociación nos encargamos de almacenarlos y ceder su uso a los estudiantes para que sea más satisfactoria su labor a la hora de preparar un examen.}

Sobre $\mathbb{R}$ se considera la topología dada por:

$$T = \{U \subset \mathbb{R} / 0 \in U \ y \ 1 \notin U \} \cup \{ \emptyset, \mathbb{R} \}$$.

\begin{enumerate}

	\item Sea $\{F_i\}_{i \in I} \subseteq C_T$ con $I$ arbitrario. ¿Es cierto que $\cup_{i \in I}F_i \in C_T$?

	\item Para cada $x \in \mathbb{R}$ obtener, si es posible, una base de entornos de $x$ en ($\mathbb{R}, T$) con un entorno.

	\item Calcular razonadamente el interior en ($\mathbb{R}, T$) de cualquier $A \subseteq \mathbb{R}$.

	\item Sea $\mathbb{I} = \mathbb{R} \textbackslash \mathbb{Q}$. ¿Es $\mathbb{I}$ denso en ($\mathbb{R}, T$)? ¿Es $\mathbb{I}$ discreto en ($\mathbb{R}, T$)?

	\item (complementario y opcional). ¿Cumple ($\mathbb{R}, T$) el IIAN?

\end{enumerate}

\end{document}

