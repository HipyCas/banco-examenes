\documentclass[fleqn]{article}
\usepackage[left=1in, right=1in, top=1in, bottom=1in]{geometry}
\usepackage[utf8]{inputenc}
\usepackage[spanish]{babel}
\usepackage{mathexam}
\usepackage{amsmath}
\usepackage{amsfonts}
\usepackage{graphicx}
\usepackage{multirow}

\ExamClass{Geometría III}
\ExamName{AMAT}
\ExamHead{10 de diciembre de 2019}

\let\ds\displaystyle

\begin{document}
\ExamInstrBox{Nombre Apellido Apellido
	\\ \\
	Parcial - Grupo B
	\\ \\
	Este examen pertenece al Banco de Exámanes de la Asociación de Estudiantes de Matemáticas de la Universidad de Granada. Si bien su autoría corresponde a los profesores ya citados, en la asociación nos encargamos de almacenarlos y ceder su uso a los estudiantes para que sea más satisfactoria su labor a la hora de preparar un examen.}

\begin{enumerate}

	\item Se considera la aplicación $f : \mathbb{R}^2 \to \mathbb{R}^2$ dada por:

	      $$f(x, y) = (1 - \frac{\sqrt{3}x}{2} + \frac{y}{2}, \frac{2}{3} + \frac{x}{2} + \frac{\sqrt{3}y}{2})$$

	      Se pide:

	      \begin{enumerate}
		      \item[(1)] Demostrar que $f$ es un movimiento rígido de $\mathbb{R}^2$, con su métrica euclídea usual.
		      \item[(2)] Calcular el vector $u_f$.
		      \item[(3)] Encontrar la forma canónica de $f$ y un sistema de referencia ortonormal donde $f$ adopte esa forma canónica.
		      \item[(4)] ¿Qué tipo de movimiento rígido es $f$? Razona la respuesta.
	      \end{enumerate}

	\item Sea $f: A \to A$ una afinidad. Demostrar:

	      \begin{itemize}
		      \item Si $f^2$ tiene un punto fijo, entonces $f$ también tiene un punto fijo.
		      \item Si $f^n$ tiene un punto fijo, para algún $n \in \mathbb{N}$, entonces $f$ también tiene un punto fijo.
	      \end{itemize}

\end{enumerate}

\end{document}

