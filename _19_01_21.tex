\documentclass[fleqn]{article}
\usepackage[left=1in, right=1in, top=1in, bottom=1in]{geometry}
\usepackage[utf8]{inputenc}
\usepackage[spanish]{babel}
\usepackage{mathexam}
\usepackage{amsmath}
\usepackage{amsfonts}
\usepackage{graphicx}
\usepackage{multirow}

\ExamClass{Topología I}
\ExamName{AMAT}
\ExamHead{19 de enero de 2021}

\let\ds\displaystyle

\begin{document}
\ExamInstrBox{Profesor
  \\ \\
  Ordinario
  \\ \\
  Este examen pertenece al Banco de Exámanes de la Asociación de Estudiantes de Matemáticas de la Universidad de Granada. Si bien su autoría corresponde a los profesores ya citados, en la asociación nos encargamos de almacenarlos y ceder su uso a los estudiantes para que sea más satisfactoria su labor a la hora de preparar un examen.}

\begin{enumerate}

  \item Para cada $\alpha \in \mathbb{R}$ denotamos $R_\alpha = \{(x, y) \in \mathbb{R}^2 / y = \alpha\}$. Se considera la topología $\mathcal{T}$ en $\mathbb{R}^2$ con base $\mathcal{B} = \{R_\alpha / \alpha \in \mathbb{R}\}$.

        \begin{enumerate}
          \item Comparar $\mathcal{T}$ con la topología usual $\mathcal{T}_u$.
          \item ¿Es $(\mathbb{R}^2, \mathcal{T})$ un espacio de Hausdorff?
          \item Calcular el cierre, el interior y la frontera de los ejes de coordenados.
          \item ¿Es cierto que todo conjunto acotado en $\mathbb{R}^2$ tiene interior vacío?
          \item Construir explícitamente un homeomorfismo $f : (\mathbb{R}^2, \mathcal{T}) \longrightarrow (\mathbb{R}^2, \mathcal{T}')$, donde $\mathcal{T}'$ es la topología en $\mathbb{R}^2$ con base $\mathcal{B}' = \{R_\alpha' / \alpha \in \mathbb{R}\}$ (aquí $R_\alpha' = \{(x, y) \in \mathbb{R}^2 / x = \alpha\}$).
          \item Probar que $A \subset \mathbb{R}^2$ es conexo en $(\mathbb{R}^2, \mathcal{T})$ si y sólo si existe $\alpha \in \mathbb{R}$ tal que $A \subset R_\alpha$. Determinar las componentes conexas de $(\mathbb{R}^2, \mathcal{T})$.
          \item Caracterizar los compactos de $(\mathbb{R}^2, \mathcal{T})$.
        \end{enumerate}

  \item Sea $f : (X, \mathcal{T}) \longrightarrow (Y, \mathcal{T}')$ una identificación. Probar que:

        \begin{enumerate}
          \item $h : (Y, \mathcal{T}') \longrightarrow (Z, \mathcal{T}'')$ es identificación si y solo si $h \circ f$ es identificación.
          \item Existe una relación de equivalencia $R$ en $X$ tal que el espacio cociente $(X/R, \mathcal{T}/R)$ es homeomorfo a $(Y, \mathcal{T}')$.
        \end{enumerate}

  \item Estudiar de forma razonada las siguientes cuestiones:

        \begin{enumerate}
          \item ¿Es cierto que la topología inducida en un subconjunto finito siempre es la discreta? ¿Y si el espacio es metrizable?
          \item Sea $(\mathbb{R}^2, \mathcal{T})$ el espacio topológico producto de dos rectas de Sorgenfrey $(\mathbb{R}, \mathcal{T}_S)$. Definimos $f : (\mathbb{R}^2, \mathcal{T}) \longrightarrow (\mathbb{R}, \mathcal{T})$ como $f(x, y) = (x -y^3)$. Analizar si $f$ es continua, abierta o cerrada.
          \item Una aplicación $f : (X, \mathcal{T}) \longrightarrow (Y, \mathcal{T}')$ es \textit{propia} si para cada $C'$ compacto en $(Y, \mathcal{T}')$ se verifica que $f^{-1}(C')$ es compacto en $(X, \mathcal{T})$. Probar que si $f$ es propia, con $(X, \mathcal{T})$ Hausdorff e $(Y, \mathcal{T}')$ compacto, entonces $f$ es continua.
        \end{enumerate}

\end{enumerate}

\end{document}